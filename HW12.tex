\documentclass{article}\usepackage[]{graphicx}\usepackage[]{xcolor}
% maxwidth is the original width if it is less than linewidth
% otherwise use linewidth (to make sure the graphics do not exceed the margin)
\makeatletter
\def\maxwidth{ %
  \ifdim\Gin@nat@width>\linewidth
    \linewidth
  \else
    \Gin@nat@width
  \fi
}
\makeatother

\definecolor{fgcolor}{rgb}{0.345, 0.345, 0.345}
\newcommand{\hlnum}[1]{\textcolor[rgb]{0.686,0.059,0.569}{#1}}%
\newcommand{\hlsng}[1]{\textcolor[rgb]{0.192,0.494,0.8}{#1}}%
\newcommand{\hlcom}[1]{\textcolor[rgb]{0.678,0.584,0.686}{\textit{#1}}}%
\newcommand{\hlopt}[1]{\textcolor[rgb]{0,0,0}{#1}}%
\newcommand{\hldef}[1]{\textcolor[rgb]{0.345,0.345,0.345}{#1}}%
\newcommand{\hlkwa}[1]{\textcolor[rgb]{0.161,0.373,0.58}{\textbf{#1}}}%
\newcommand{\hlkwb}[1]{\textcolor[rgb]{0.69,0.353,0.396}{#1}}%
\newcommand{\hlkwc}[1]{\textcolor[rgb]{0.333,0.667,0.333}{#1}}%
\newcommand{\hlkwd}[1]{\textcolor[rgb]{0.737,0.353,0.396}{\textbf{#1}}}%
\let\hlipl\hlkwb

\usepackage{framed}
\makeatletter
\newenvironment{kframe}{%
 \def\at@end@of@kframe{}%
 \ifinner\ifhmode%
  \def\at@end@of@kframe{\end{minipage}}%
  \begin{minipage}{\columnwidth}%
 \fi\fi%
 \def\FrameCommand##1{\hskip\@totalleftmargin \hskip-\fboxsep
 \colorbox{shadecolor}{##1}\hskip-\fboxsep
     % There is no \\@totalrightmargin, so:
     \hskip-\linewidth \hskip-\@totalleftmargin \hskip\columnwidth}%
 \MakeFramed {\advance\hsize-\width
   \@totalleftmargin\z@ \linewidth\hsize
   \@setminipage}}%
 {\par\unskip\endMakeFramed%
 \at@end@of@kframe}
\makeatother

\definecolor{shadecolor}{rgb}{.97, .97, .97}
\definecolor{messagecolor}{rgb}{0, 0, 0}
\definecolor{warningcolor}{rgb}{1, 0, 1}
\definecolor{errorcolor}{rgb}{1, 0, 0}
\newenvironment{knitrout}{}{} % an empty environment to be redefined in TeX

\usepackage{alltt}
\usepackage[margin=1.0in]{geometry} % To set margins
\usepackage{amsmath}  % This allows me to use the align functionality.
                      % If you find yourself trying to replicate
                      % something you found online, ensure you're
                      % loading the necessary packages!
\usepackage{amsfonts} % Math font
\usepackage{fancyvrb}
\usepackage{hyperref} % For including hyperlinks
\usepackage[shortlabels]{enumitem}% For enumerated lists with labels specified
                                  % We had to run tlmgr_install("enumitem") in R
\usepackage{float}    % For telling R where to put a table/figure
\usepackage{natbib}        %For the bibliography
\bibliographystyle{apalike}%For the bibliography
\IfFileExists{upquote.sty}{\usepackage{upquote}}{}
\begin{document}


\begin{enumerate}
%%%%%%%%%%%%%%%%%%%%%%%%%%%%%%%%%%%%%%%%%%%%%%%%%%%%%%%%%%%%%%%%%%%%%%%%%%%%%%%%
%%%%%%%%%%%%%%%%%%%%%%%%%%%%%%%%%%%%%%%%%%%%%%%%%%%%%%%%%%%%%%%%%%%%%%%%%%%%%%%%
% Question 1
%%%%%%%%%%%%%%%%%%%%%%%%%%%%%%%%%%%%%%%%%%%%%%%%%%%%%%%%%%%%%%%%%%%%%%%%%%%%%%%%
%%%%%%%%%%%%%%%%%%%%%%%%%%%%%%%%%%%%%%%%%%%%%%%%%%%%%%%%%%%%%%%%%%%%%%%%%%%%%%%%
\item A group of researchers is running an experiment over the course of 30 months, 
with a single observation collected at the end of each month. Let $X_1, ..., X_{30}$
denote the observations for each month. From prior studies, the researchers know that
\[X_i \sim f_X(x),\]
but the mean $\mu_X$ is unknown, and they wish to conduct the following test
\begin{align*}
H_0&: \mu_X = 0\\
H_a&: \mu_X > 0.
\end{align*}
At month $k$, they have accumulated data $X_1, ..., X_k$ and they have the 
$t$-statistic
\[T_k = \frac{\bar{X} - 0}{S_k/\sqrt{n}}.\]
The initial plan was to test the hypotheses after all data was collected (at the 
end of month 30), at level $\alpha=0.05$. However, conducting the experiment is 
expensive, so the researchers want to ``peek" at the data at the end of month 20 
to see if they can stop it early. That is, the researchers propose to check 
whether $t_{20}$ provides statistically discernible support for the alternative. 
If it does, they will stop the experiment early and report support for the 
researcher's alternative hypothesis. If it does not, they will continue to month 
30 and test whether $t_{30}$ provides statistically discernible support for the
alternative.

\begin{enumerate}
  \item What values of $t_{20}$ provide statistically discernible support for the
  alternative hypothesis?
  \textbf{Solution:} We can calculate the value of $t_{20}$ that provide discernible support for the alternative by finding the t-value that is where the $\alpha = 0.05$ cutoff is. For $t_{20}$, that value is 1.7291, so any values greater than or equal to that provide support for the alternative.
\begin{knitrout}\scriptsize
\definecolor{shadecolor}{rgb}{0.969, 0.969, 0.969}\color{fgcolor}\begin{kframe}
\begin{alltt}
\hlcom{#values of t20 that provide support}
\hldef{alpha} \hlkwb{=} \hlnum{0.05}
\hldef{n} \hlkwb{=} \hlnum{20}  \hlcom{# sample size}
\hldef{df} \hlkwb{<-} \hldef{n} \hlopt{-} \hlnum{1}  \hlcom{# degrees of freedom}
\hldef{t_crit_20} \hlkwb{<-} \hlkwd{qt}\hldef{(}\hlnum{1} \hlopt{-} \hldef{alpha,} \hlkwc{df} \hldef{= df)}
\end{alltt}
\end{kframe}
\end{knitrout}
  \item What values of $t_{30}$ provide statistically discernible support for the
  alternative hypothesis?
  \textbf{Solution:} Using the same approach as for $t_{20}$, but replacing n=20 with n=30, we find that the $\alpha = 0.05$ t-value is 1.6991, and consequently any values greater than this provide statistically discernible support for the alternative.
  \item Suppose $f_X(x)$ is a Laplace distribution with $a=0$ and $b=4.0$.
  Conduct a simulation study to assess the Type I error rate of this approach.\\
  \textbf{Note:} You can use the \texttt{rlaplace()} function from the \texttt{VGAM}
  package for \texttt{R} \citep{VGAM}.
  
  
  \textbf{Solution:} To conduct the simulation study, we take $n$ number of observations ($X$) from the Laplace distribution $a = 0$ and $b = 4.0$. For this set of $X$, we then calculate the t-statistic using the formula provided above, as we can calculate the mean and standard deviation for the simulated observations and we also know $n$ (either 20 or 30). Then, we compare this t-statistic to the critical points we calculated in a) and b), where if the simulated t-statistic is greater than the critical point, we reject the null hypothesis. By counting the number of rejections in the 10000 samples, and then dividing it by the total number of samples (10000), we can calculate the Type I error. We find that for $t_{20}$, the $Type I error = 0.0510$ and for $t_{20}$ $Type I error = 0.0494$. These values are close to the expected $\alpha = 0.05$. 
\begin{knitrout}\scriptsize
\definecolor{shadecolor}{rgb}{0.969, 0.969, 0.969}\color{fgcolor}\begin{kframe}
\begin{alltt}
\hlcom{#Simulation Study for Type I error}
\hldef{R} \hlkwb{=} \hlnum{10000} \hlcom{#number of sims}
\hldef{alpha} \hlkwb{=} \hlnum{0.05}
\hldef{set.seed} \hlkwb{=} \hlnum{151}
\hldef{a} \hlkwb{=} \hlnum{0}
\hldef{b} \hlkwb{=} \hlnum{4}
\hldef{n} \hlkwb{=} \hlnum{20} \hlcom{#t_20}
\hldef{rejects} \hlkwb{=} \hlnum{0} \hlcom{#will add to this for each time we reject H0}

\hlkwa{for} \hldef{(i} \hlkwa{in} \hlnum{1}\hlopt{:}\hldef{R) \{}
\hldef{x} \hlkwb{<-} \hlkwd{rlaplace}\hldef{(n,} \hlkwc{location} \hldef{= a,} \hlkwc{scale} \hldef{= b)}
\hlcom{#take observation (x) from the Laplace dist. n times}
\hldef{t} \hlkwb{=} \hlkwd{mean}\hldef{(x)} \hlopt{/} \hldef{(}\hlkwd{sd}\hldef{(x)} \hlopt{/} \hlkwd{sqrt}\hldef{(n))} \hlcom{#T-stat formula}

\hlcom{#Compare to critical point calculated in a) and b)}
\hlkwa{if} \hldef{(t} \hlopt{>} \hldef{t_crit_20)\{}
  \hldef{rejects} \hlkwb{=} \hldef{rejects} \hlopt{+} \hlnum{1}
\hldef{\}}
\hldef{\}}
\hldef{TypeI_20} \hlkwb{=} \hldef{rejects}\hlopt{/}\hldef{R}

\hldef{n} \hlkwb{=} \hlnum{30} \hlcom{#t_30}
\hldef{rejects} \hlkwb{=} \hlnum{0} \hlcom{#will add to this for each time we reject H0}

\hlkwa{for} \hldef{(i} \hlkwa{in} \hlnum{1}\hlopt{:}\hldef{R) \{}
\hldef{x} \hlkwb{<-} \hlkwd{rlaplace}\hldef{(n,} \hlkwc{location} \hldef{= a,} \hlkwc{scale} \hldef{= b)}
\hlcom{#take observation (x) from the Laplace dist. n times}
\hldef{t} \hlkwb{=} \hlkwd{mean}\hldef{(x)} \hlopt{/} \hldef{(}\hlkwd{sd}\hldef{(x)} \hlopt{/} \hlkwd{sqrt}\hldef{(n))} \hlcom{#T-stat formula}

\hlcom{#Compare to critical point calculated in a) and b)}
\hlkwa{if} \hldef{(t} \hlopt{>} \hldef{t_crit_30)\{}
  \hldef{rejects} \hlkwb{=} \hldef{rejects} \hlopt{+} \hlnum{1}
\hldef{\}}
\hldef{\}}
\hldef{TypeI_30} \hlkwb{=} \hldef{rejects}\hlopt{/}\hldef{R}
\end{alltt}
\end{kframe}
\end{knitrout}
  \item \textbf{Optional Challenge:} Can you find a value of $\alpha<0.05$ that yields a 
  Type I error rate of 0.05?
\end{enumerate}
%%%%%%%%%%%%%%%%%%%%%%%%%%%%%%%%%%%%%%%%%%%%%%%%%%%%%%%%%%%%%%%%%%%%%%%%%%%%%%%%
%%%%%%%%%%%%%%%%%%%%%%%%%%%%%%%%%%%%%%%%%%%%%%%%%%%%%%%%%%%%%%%%%%%%%%%%%%%%%%%%
% Question 2
%%%%%%%%%%%%%%%%%%%%%%%%%%%%%%%%%%%%%%%%%%%%%%%%%%%%%%%%%%%%%%%%%%%%%%%%%%%%%%%%
%%%%%%%%%%%%%%%%%%%%%%%%%%%%%%%%%%%%%%%%%%%%%%%%%%%%%%%%%%%%%%%%%%%%%%%%%%%%%%%%
  \item Perform a simulation study to assess the robustness of the $T$ test. 
  Specifically, generate samples of size $n=15$ from the Beta(10,2), Beta(2,10), 
  and Beta(10,10) distributions and conduct the following hypothesis tests against 
  the actual mean for each case (e.g., $\frac{10}{10+2}$, $\frac{2}{10+2}$, and 
  $\frac{10}{10+10}$). 
  \begin{enumerate}
    \item What proportion of the time do we make an error of Type I for a
    left-tailed test?
    
    \textbf{Solution:} We conducted the simulation in the same way as for calculating the Type I error for the Laplace distribution. The code was repeated the same for parts b) and c) as in part a), where just $\alpha$ and $\beta$ for the Beta distribution were changed. All the results can be seen in the table after part c. 
    
\begin{knitrout}\scriptsize
\definecolor{shadecolor}{rgb}{0.969, 0.969, 0.969}\color{fgcolor}\begin{kframe}
\begin{alltt}
\hlcom{##################################}
\hlcom{# Question 2}
\hlcom{##################################}
\hldef{R} \hlkwb{=} \hlnum{10000}
\hldef{n} \hlkwb{=} \hlnum{15}
\hlcom{#critical t-point which is a = 0.05}
\hldef{alpha} \hlkwb{<-} \hlnum{0.05}
\hldef{df} \hlkwb{<-} \hldef{n} \hlopt{-} \hlnum{1}  \hlcom{# degrees of freedom}

\hldef{t15_rtail} \hlkwb{=} \hlkwd{qt}\hldef{(}\hlnum{1} \hlopt{-} \hldef{alpha,} \hlkwc{df} \hldef{= df)}  \hlcom{#right-tailed text}
\hldef{t15_ltail} \hlkwb{=} \hlkwd{qt}\hldef{(alpha,} \hlkwc{df} \hldef{= df)} \hlcom{#left-tailed test}
\hldef{t15_2tail_low} \hlkwb{=} \hlkwd{qt}\hldef{(alpha}\hlopt{/}\hlnum{2}\hldef{,} \hlkwc{df} \hldef{= df)} \hlcom{# left side of two tailed}
\hldef{t15_2tail_high} \hlkwb{=} \hlkwd{qt}\hldef{(}\hlnum{1} \hlopt{-} \hldef{alpha}\hlopt{/}\hlnum{2}\hldef{,} \hlkwc{df} \hldef{= df)} \hlcom{#right side of two tailed}

\hlcom{##########}
\hlcom{#Beta(10,2)}
\hlcom{##########}
\hldef{a} \hlkwb{=} \hlnum{10}
\hldef{b} \hlkwb{=} \hlnum{2}
\hldef{mu} \hlkwb{=} \hldef{a}\hlopt{/}\hldef{(a}\hlopt{+}\hldef{b)}

\hldef{left_reject} \hlkwb{=} \hlnum{0}
\hldef{right_reject} \hlkwb{=} \hlnum{0}
\hldef{twotail_reject} \hlkwb{=} \hlnum{0}

\hlkwa{for} \hldef{(i} \hlkwa{in} \hlnum{1}\hlopt{:}\hldef{R)\{}
  \hldef{x} \hlkwb{=} \hlkwd{rbeta}\hldef{(}\hlkwc{n} \hldef{= n,} \hlkwc{shape1} \hldef{= a,} \hlkwc{shape2} \hldef{= b)}
  \hlcom{#left-tailed test}
  \hldef{t_left} \hlkwb{=} \hlkwd{t.test}\hldef{(x,} \hlkwc{alternative} \hldef{=} \hlsng{"less"}\hldef{,} \hlkwc{mu} \hldef{= mu)}\hlopt{$}\hldef{statistic}
  \hlkwa{if} \hldef{(t_left} \hlopt{<} \hldef{t15_ltail)\{}
    \hldef{left_reject} \hlkwb{=} \hldef{left_reject} \hlopt{+} \hlnum{1}
  \hldef{\}}
  \hlcom{#right-tailed test}
  \hldef{t_right} \hlkwb{=} \hlkwd{t.test}\hldef{(x,} \hlkwc{alternative} \hldef{=} \hlsng{"greater"}\hldef{,} \hlkwc{mu} \hldef{= mu)}\hlopt{$}\hldef{statistic}
  \hlkwa{if} \hldef{(t_right} \hlopt{>} \hldef{t15_rtail)\{}
    \hldef{right_reject} \hlkwb{=} \hldef{right_reject} \hlopt{+} \hlnum{1}
  \hldef{\}}
  \hlcom{#two-tail test}
  \hldef{t_twotail} \hlkwb{=} \hlkwd{t.test}\hldef{(x,} \hlkwc{alternative} \hldef{=} \hlsng{"two.sided"}\hldef{,} \hlkwc{mu} \hldef{= mu)}\hlopt{$}\hldef{statistic}
  \hlkwa{if} \hldef{(t_twotail} \hlopt{<} \hldef{t15_2tail_low} \hlopt{|} \hldef{t_twotail} \hlopt{>} \hldef{t15_2tail_high)\{}
    \hldef{twotail_reject} \hlkwb{=} \hldef{twotail_reject} \hlopt{+} \hlnum{1}
  \hldef{\}}
\hldef{\}}


\hlcom{#Calculating the Type I error}
\hldef{TypeI_left_102} \hlkwb{=} \hldef{left_reject}\hlopt{/}\hldef{R}
\hldef{TypeI_right_102} \hlkwb{=} \hldef{right_reject}\hlopt{/}\hldef{R}
\hldef{TypeI_2tail_102} \hlkwb{=} \hldef{twotail_reject}\hlopt{/}\hldef{R}
\end{alltt}
\end{kframe}
\end{knitrout}
    \item What proportion of the time do we make an error of Type I for a
    right-tailed test?
    \item What proportion of the time do we make an error of Type I for a
    two-tailed test?
    
    % latex table generated in R 4.4.2 by xtable 1.8-4 package
% Wed Apr 23 16:22:08 2025
\begin{table}[ht]
\centering
\begin{tabular}{rlrrr}
  \hline
 & Beta & left\_tailed & right\_tailed & two\_tailed \\ 
  \hline
1 & Beta(10,2) & 0.03 & 0.08 & 0.06 \\ 
  2 & Beta(2,10) & 0.08 & 0.03 & 0.06 \\ 
  3 & Beta(10,10) & 0.05 & 0.05 & 0.05 \\ 
   \hline
\end{tabular}
\end{table}
    
    \item How does skewness of the underlying population distribution effect
    Type I error across the test types?
    
    \textbf{Solution:} The Beta(10,2) distribution is left skewed. As such, it has a lower Type I error for the left-tailed distribution and a higher error for the right-tailed.  The inverse is true for the Beta(2,10) as it is right skewed. Finally, as Beta(10,10) is symmetric, it makes sense that the left and right-tailed tests result in the same Type I error, which would also match the two-tailed test.
  \end{enumerate}
%%%%%%%%%%%%%%%%%%%%%%%%%%%%%%%%%%%%%%%%%%%%%%%%%%%%%%%%%%%%%%%%%%%%%%%%%%%%%%%%
%%%%%%%%%%%%%%%%%%%%%%%%%%%%%%%%%%%%%%%%%%%%%%%%%%%%%%%%%%%%%%%%%%%%%%%%%%%%%%%%
% End Document
%%%%%%%%%%%%%%%%%%%%%%%%%%%%%%%%%%%%%%%%%%%%%%%%%%%%%%%%%%%%%%%%%%%%%%%%%%%%%%%%
%%%%%%%%%%%%%%%%%%%%%%%%%%%%%%%%%%%%%%%%%%%%%%%%%%%%%%%%%%%%%%%%%%%%%%%%%%%%%%%%
\end{enumerate}
\bibliography{bibliography}
\end{document}
